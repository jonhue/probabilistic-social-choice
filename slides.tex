\documentclass{beamer}
%
% Choose how your presentation looks.
%
% For more themes, color themes and font themes, see:
% http://deic.uab.es/~iblanes/beamer_gallery/index_by_theme.html
%
\mode<presentation>
{
    \usetheme{default}      % or try Darmstadt, Madrid, Warsaw, ...
    \usecolortheme{default} % or try albatross, beaver, crane, ...
    \usefonttheme{default}  % or try serif, structurebold, ...
    \setbeamertemplate{navigation symbols}{}
    \setbeamertemplate{caption}[numbered]
    \setbeamertemplate{footline}[frame number]
    \setbeamertemplate{itemize items}[circle]
    \setbeamertemplate{theorems}[]
    \setbeamercolor*{structure}{bg=white,fg=blue}
    \setbeamerfont{block title}{size=\normalsize}
    \setbeamertemplate{frametitle continuation}{}
}

\newtheorem{proposition}[theorem]{Proposition}
\theoremstyle{definition}
\newtheorem{algorithm}[theorem]{Algorithm}
\newtheorem{idea}[theorem]{Idea}

\usepackage[english]{babel}
\usepackage[utf8]{inputenc}
\usepackage[T1]{fontenc}
\usepackage{aligned-overset}
\usepackage{alltt}
\usepackage{amsmath}
\usepackage{csquotes}
\usepackage{multicol}
\usepackage{stmaryrd}
\usepackage{tabularx}
\usepackage{caption}
\usepackage{nicefrac}

\captionsetup{labelformat=empty}

\renewcommand\tabularxcolumn[1]{m{#1}}
\newcolumntype{R}{>{\raggedleft\arraybackslash}X}

\def\code#1{\texttt{\frenchspacing#1}}
\def\padding{\vspace{0.5cm}}
\def\spadding{\vspace{0.25cm}}
\def\b{\textcolor{blue}}
\def\r{\textcolor{red}}
\def\o{\textcolor{gray}}
\def\g#1{{\usebeamercolor[fg]{block title example}{#1}}}

% fix for \pause in align
\makeatletter
\let\save@measuring@true\measuring@true
\def\measuring@true{%
    \save@measuring@true
    \def\beamer@sortzero##1{\beamer@ifnextcharospec{\beamer@sortzeroread{##1}}{}}%
    \def\beamer@sortzeroread##1<##2>{}%
    \def\beamer@finalnospec{}%
}
\makeatother

\usepackage[sorting=ynt,style=authoryear]{biblatex}
\addbibresource{sources.bib}

\title[Probabilistic Social Choice]{Probabilistic Social Choice \\ \small{(\cite{Brandl2015})}}
\author{Jonas Hübotter}
\date{November 23, 2021}

\begin{document}

\begin{frame}
    \titlepage
\end{frame}

% Uncomment these lines for an automatically generated outline.
% \begin{frame}[allowframebreaks]{Outline}
%     \tableofcontents[subsubsectionstyle=hide]
% \end{frame}
% \AtBeginSection[]
%   {
%      \begin{frame}
%      \frametitle{Plan}
%      \tableofcontents[currentsection, sectionstyle=show/hide, hideothersubsections]
%      \end{frame}
%   }

\section{Introduction}

\begin{frame}
\begin{center}
Ava, Ben, and Chloe want to go for lunch.\pause \\[10pt]

Ava wants to go to Klara's Kitchen $\succ$ Mensa $\succ$ foodLAB.\par
Ben wants to go to Mensa $\succ$ foodLAB $\succ$ Klara's Kitchen.\par
Chloe wants to go to foodLAB $\succ$ Klara's Kitchen $\succ$ Mensa.\pause \\[10pt]

So where to go?\pause \\[25pt]

\large{Aggregating the preferences of individuals\par is the \b{problem of social choice}.}
\end{center}
\end{frame}

\section{Deterministic Social Choice}

\begin{frame}{Framework of Social Choice}
Given \begin{itemize}
    \item an \b{electorate} \g{(Ava, Ben, and Chloe)}\pause,
    \item a set of \b{alternatives} $A\ \g{= \{\text{foodLAB}, \text{Klara's Kitchen}, \text{Mensa}\}}$\pause,
    \item a \b{preference profile} $R$ over $A$\pause \\[8pt]
        \begin{center}\begin{tabular}{ c|c|c }
         $\nicefrac{1}{3}$ & $\nicefrac{1}{3}$ & $\nicefrac{1}{3}$ \\
         \hline\hline
         K & M & F \\
         M & F & K \\
         F & K & M \\
        \end{tabular}\end{center}
\end{itemize}\vspace{8pt}\pause

find a \b{social choice function} $f : \mathcal{R}_A\only<5->{\footnote{$\mathcal{R}_A = \Delta(\mathcal{L}(A))$ is the set of all preference profiles over A.}} \to A$ that returns a single alternative\pause,

\g{i.e., identifies a place for lunch that is in $A$.}
\end{frame}

\begin{frame}{Framework of Social Choice}

\begin{center}\begin{tabular}{ c|c|c }
 $\nicefrac{1}{3}$ & $\nicefrac{1}{3}$ & $\nicefrac{1}{3}$ \\
 \hline\hline
 K & M & F \\
 M & F & K \\
 F & K & M \\
\end{tabular}\end{center}\vspace{10pt}

$R(a \succ b)$ is the fraction of the electorate that prefers $a$ to $b$.\pause \\[10pt]

\begin{center}
\g{$R(K \succ M)\pause = \nicefrac{2}{3}$.}
\end{center}
\end{frame}

\subsection{Condorcet Consistency}

\begin{frame}{Framework of Social Choice}
\begin{center}
We do not just want \emph{some} social choice function,\par
we want a \emph{good} social choice function.\pause \\[10pt]
We need an \b{axiomatic characterization}.
\end{center}\pause

\begin{block}{Condorcet consistency}
If $R(a \succ b) > \nicefrac{1}{2}$ for any $b \in A \setminus \{a\}$\pause, then $a$ is called a \b{Condorcet~winner}.\par\pause
$f$ is \b{Condorcet consistent} if it always chooses the Condorcet winner provided one exists.
\end{block}

% \begin{block}{Weak Pareto Efficiency}
% If $R(\text{Klara's Kitchen} \succ \text{Mensa})\only<4->{\footnote{$R(a \succ b)$ is the fraction of the electorate that prefers $a$ to $b$.}} = 1$,\pause\ then $f(R) \neq \text{Mensa}$.\par\pause
% \g{We would be better off collectively by going to Klara's Kitchen.}
% \end{block}\pause

% \begin{block}{Monotonicty}
% If $R(\text{Klara's Kitchen} \succ b) \leq R'(\text{Klara's Kitchen} \succ b)$,\pause\par
% then $f(R) = \text{Klara's Kitchen}$ implies $f(R') = \text{Klara's Kitchen}$.\par\pause
% \g{When we decide to go to Klara's Kitchen and the support for going there is increased, we should still go there.}
% \end{block}
\end{frame}

\begin{frame}{Framework of Social Choice}
\begin{center}\begin{tabular}{ c|c|c }
 $\nicefrac{1}{3}$ & $\nicefrac{1}{3}$ & $\nicefrac{1}{3}$ \\
 \hline\hline
 K & M & F \\
 M & F & K \\
 F & K & M \\
\end{tabular}\end{center}\pause

The preferences of Ava, Ben, and Chloe do not yield a Condorcet~winner (known as the Condorcet paradox).\pause \\[20pt]

\begin{center}
\r{Many commonly used social choice functions are Condorcet~inconsistent.}\par\pause
\small{(plurality, two-round runoff, instant runoff, Borda count)}
\end{center}\pause\vspace{10pt}

$\implies$ deterministic social choice functions are problematic
\end{frame}

\section{Probabilistic Social Choice}

\begin{frame}{Probabilistic Social Choice}
Instead of returning a single alternative, a \b{probabilistic social choice function} returns probability distributions (\b{lotteries}) over alternatives.\pause \\[10pt]

Formally, $f : \mathcal{R}_A \to \mathcal{P}(\Delta(A))$ where $\Delta(A)$ is the set of lotteries over $A$.\pause

\begin{example}[random dictatorship]
Choose a preference relation among all preference relations uniformly at random, and choose its most-preferred alternative.

\begin{center}\begin{tabular}{ c|c|c }
 $\nicefrac{1}{3}$ & $\nicefrac{1}{3}$ & $\nicefrac{1}{3}$ \\
 \hline\hline
 \r{K} & \r{M} & \r{F} \\
 \o{M} & \o{F} & \o{K} \\
 \o{F} & \o{K} & \o{M} \\
\end{tabular}\end{center}
\end{example}
\end{frame}

\begin{frame}{Probabilistic Social Choice}
\begin{block}{Requirements}
\begin{itemize}\pause
    \item when Ava goes to lunch on her own, she should choose Klara's~Kitchen with probability $1$ (\b{unanimity})\pause
    \item when $f$ returns multiple lotteries for a preference profile, there is an "arbitrarily close" preference profile that only yields a single lottery (\b{decisiveness})
\end{itemize}
\end{block}\pause
$\implies$ deterministic social choice functions are a special case of probabilistic social choice functions which only return \emph{degenerate} lotteries (i.e., lotteries that put all probability mass on a single alternative).
\end{frame}

\subsection{Population Consistency}

\begin{frame}{Consistency Axioms}
\begin{columns}
\begin{column}{0.25\textwidth}
\begin{center}\begin{table}\begin{tabular}{ c|c }
 $\nicefrac{1}{2}$ & $\nicefrac{1}{2}$ \\
 \hline\hline
 K & M \\
 M & F \\
 F & K \\
\end{tabular}\caption{$R'$}\end{table}\end{center}
\end{column}
\begin{column}{0.25\textwidth}
\begin{center}\begin{table}\begin{tabular}{ c|c }
 $\nicefrac{1}{2}$ & $\nicefrac{1}{2}$ \\
 \hline\hline
 K & M \\
 F & F \\
 M & K \\
\end{tabular}\caption{$R''$}\end{table}\end{center}
\end{column}\pause
\begin{column}{0.5\textwidth}
\begin{center}\begin{table}\begin{tabular}{ c|c|c }
 $\nicefrac{1}{4}$ & $\nicefrac{1}{4}$ & $\nicefrac{1}{2}$ \\
 \hline\hline
 K & K & M \\
 M & F & F \\
 F & M & K \\
\end{tabular}\caption{$\nicefrac{1}{2} R' + \nicefrac{1}{2} R'' = R$}\end{table}\end{center}
\end{column}
\end{columns}\pause

\g{When $\nicefrac{1}{2}K + \nicefrac{1}{2}M$ is chosen by $R'$ and $R''$, then it must also be chosen by $R$.}\pause

\begin{block}{Population Consistency}
When two separate electorates $R'$ and $R''$ agree on a lottery, this lottery must also be chosen by a combination of the electorates.

$\implies$ consistency with respect to a variable electorate
\end{block}
\end{frame}

\subsection{Cloning Consistency}

\begin{frame}{Consistency Axioms}
Ben \emph{really} wants to go to Mensa so he devises an evil plan to confuse his friends.\pause\par
Instead of proposing to go to Mensa, he introduces separate \b{clones} $B = \{M1, M2, \dots\}$ for table 1 at Mensa, table 2 at Mensa, ...\pause

\begin{columns}
\begin{column}{0.33\textwidth}
\begin{center}\begin{table}\begin{tabular}{ c|c|c }
 $\nicefrac{1}{3}$ & $\nicefrac{1}{3}$ & $\nicefrac{1}{3}$ \\
 \hline\hline
 K & M1 & F \\
 M1 & M2 & K \\
 M2 & F & M2 \\
 F & K & M1 \\
\end{tabular}\caption{$R$}\end{table}\end{center}
\end{column}\pause
\begin{column}{0.33\textwidth}
\begin{center}\begin{table}\begin{tabular}{ c|c|c }
 $\nicefrac{1}{3}$ & $\nicefrac{1}{3}$ & $\nicefrac{1}{3}$ \\
 \hline\hline
 K & M1 & F \\
 M1 & F & K \\
 F & K & M1 \\
\end{tabular}\caption{$R|_{A'}$ \\ $A' = \{F, M1, K\}$}\end{table}\end{center}
\end{column}
\begin{column}{0.33\textwidth}
\begin{center}\begin{table}\begin{tabular}{ c|c }
 $\nicefrac{2}{3}$ & $\nicefrac{1}{3}$ \\
 \hline\hline
 M1 & M2 \\
 M2 & M1 \\
\end{tabular}\caption{$R|_B$ \\ $B = \{M1, M2\}$}\end{table}\end{center}
\end{column}
\end{columns}\pause

\begin{block}{Cloning Consistency}
We should assign the same probability to $M$ independently of the number of clones and the internal relationship between these clones.

$\implies$ consistency with respect to components of \emph{similar} alternatives
\end{block}
\end{frame}

\subsection{Composition Consistency}

\begin{frame}{Consistency Axioms}
\begin{columns}
\begin{column}{0.33\textwidth}
\begin{center}\begin{table}\begin{tabular}{ c|c|c }
 $\nicefrac{1}{3}$ & $\nicefrac{1}{3}$ & $\nicefrac{1}{3}$ \\
 \hline\hline
 K & M1 & F \\
 M1 & M2 & K \\
 M2 & F & M2 \\
 F & K & M1 \\
\end{tabular}\caption{$R$}\end{table}\end{center}
\end{column}
\begin{column}{0.33\textwidth}
\begin{center}\begin{table}\begin{tabular}{ c|c|c }
 $\nicefrac{1}{3}$ & $\nicefrac{1}{3}$ & $\nicefrac{1}{3}$ \\
 \hline\hline
 K & M1 & F \\
 M1 & F & K \\
 F & K & M1 \\
\end{tabular}\caption{$R|_{A'}$ \\ $A' = \{F, M1, K\}$}\end{table}\end{center}
\end{column}
\begin{column}{0.33\textwidth}
\begin{center}\begin{table}\begin{tabular}{ c|c }
 $\nicefrac{2}{3}$ & $\nicefrac{1}{3}$ \\
 \hline\hline
 M1 & M2 \\
 M2 & M1 \\
\end{tabular}\caption{$R|_B$ \\ $B = \{M1, M2\}$}\end{table}\end{center}
\end{column}
\end{columns}

\g{$f$ should assign more probability to $M1$ than to $M2$.}\pause

\begin{block}{Composition Consistency (extending cloning consistency)}
The probability of clones must be directly proportional to the probabilities that $f$ assigns to these alternatives for $R|_B$.
\end{block}
\end{frame}

\section{Results}

\begin{frame}{Results}
Deterministic social choice functions \r{cannot} satisfy \begin{itemize}
    \item Condorcet consistency and population consistency\pause
    \item population consistency and composition consistency
\end{itemize}\pause\vspace{10pt}

What about probabilistic social choice functions?
\begin{itemize}
    \item random dictatorships satisfy population consistency and cloning consistency\pause
    \item \emph{maximal lotteries} satisfy population consistency and composition consistency (and Condorcet consistency)\par(\cite{Brandl2015})
\end{itemize}
\end{frame}

\section{Interpretation}

\begin{frame}{Interpretation}
\blockquote{Sometimes, in fact, we want to exclude reasons from decision‐making entirely. This is what lotteries can do.}\par(\cite{Stone2011})\pause\vspace{10pt}

This does not entail that decisions must lack rationale, but rather that the aggregation happens randomly and as such is harder to exploit.\pause \\[20pt]

Amidst which circumstances are random choices useful?\pause
\begin{itemize}
    \item depending on the risk aversion of voters, and
    \item the effective degree of randomness.
\end{itemize}
\end{frame}

\begin{frame}{Acceptability of Lotteries}
\begin{block}{Risk Aversion}
Typically, the risk aversion depends on
\begin{itemize}
    \item the stakes of the decision\pause, and
    \item the frequency of preference aggregation.
\end{itemize}\pause

If preferences are aggregated rarely the law of large numbers does not apply and a risk-averse voter might prefer a sure outcome to a random outcome even if in expectation she prefers the random outcome.
\end{block}\pause\vspace{10pt}

In the Athenian democracy, lotteries were a principal component. The process of selecting candidates by a lottery for a political office is also known as \emph{sortition} or \emph{selection by lot}.
\end{frame}

\begin{frame}{Acceptability of Lotteries}
\begin{block}{Effective Degree of Randomness}
What to do about ties? Are ties rare or common?\pause \\[15pt]

\g{Condorcet view:} we encounter a tie iff there is no Condorcet~winner, i.e., no unequivocal winner.\pause \\[5pt]
It can be shown that Condorcet's view of equivocality and the randomness of maximal lotteries coincide.\footnote{Maximal lotteries are degenerate iff there is a Condercet winner.}\pause \\[15pt]

Empirically, there often exists a Condorcet winner. If there exists no Condorcet winner, the support of maximal lotteries is typically small compared to the number of alternatives.
\end{block}
\end{frame}

\begin{frame}{Acceptability of Lotteries}
\blockquote{A political system, such as the one in the United States, which
 rules out lotteries might lead to dominance by the center, even if
 the right and left could put together a majority coalition that would
 prefer a lottery on the extremes to a middle outcome. (\cite{Zeckhauser1969})}\pause

In general, majority rule without lotteries corrects towards the median choice.\par
This can be considered inefficient or desirable. The judgement may depend on the setting.
\end{frame}

\begin{frame}{An Alternative Interpretation}
We have been thinking about lotteries as probability distributions over alternatives, used to obtain a single (or multiple) choices.\pause

\begin{example}
Consider a firm that asks its employees to spend a fixed budget on non-profits of their choice. How can we come up with a fair allocation of the budget?
\end{example}\pause

\begin{center}
    \large{Probabilities can equivalently be interpreted as fractions of a divisible resource such as money or time.}
\end{center}
\end{frame}

\section{Discussion}

\begin{frame}
\large{\begin{center}
\g{Thank you for your attention} \\ [25pt]

How often do preferences have to be (repeatedly) aggregated until lotteries become acceptable? \\[10pt]

Is randomness in decision making at all acceptable?\par Is it desirable?
\end{center}}
\end{frame}

\begin{frame}[allowframebreaks]{References}
\nocite{*}
\printbibliography
\end{frame}

\begin{frame}{The Inefficiency of Majority Rule without Lotteries}
\blockquote{The 101 Club must choose a single form of entertainment for
 all club members. The membership rolls contain fifty football
 fanatics, fifty ballet aficionados, and a single lover of musical
 comedy. For the footballers the musical is almost as bad as the
 ballet. For the ballet enthusiasts the musical is little better than
 football. By majority rule, using pairwise comparisons without
 lotteries, the musical will be chosen. If lotteries were permitted, a
 fifty-fifty football-ballet lottery would defeat the musical by any
 required plurality up to 100 out of 101. (\cite{Zeckhauser1969})}
\end{frame}

\end{document}
